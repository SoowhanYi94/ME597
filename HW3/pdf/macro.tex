\usepackage{amsmath,amsfonts,amsthm,amssymb,amsopn,bm,mathtools}
\usepackage[margin=.9in]{geometry}
\usepackage{graphicx}
\usepackage{url}
% \usepackage[hidelinks]{hyperref}
\usepackage[usenames,dvipsnames]{color}
\usepackage{fancyhdr}
\usepackage{multirow}
\usepackage{mdframed}
\usepackage{ifthen}

% calculus
% the differential operator
\makeatletter
\providecommand*{\diff}
{\@ifnextchar^{\DIfF}{\DIfF^{}}}
\def\DIfF^#1{
	\mathop{\mathrm{\mathstrut d}}
	\nolimits^{#1}\gobblespace}
\def\gobblespace{
	\futurelet\diffarg\opspace}
\def\opspace{
	\let\DiffSpace\!
	\ifx\diffarg(
	\let\DiffSpace\relax
	\else
	\ifx\diffarg[
	\let\DiffSpace\relax
	\else
	\ifx\diffarg\{
	\let\DiffSpace\relax
	\fi\fi\fi\DiffSpace}

\providecommand*{\pdiff}
{\@ifnextchar^{\pDIfF}{\pDIfF^{}}}
\def\pDIfF^#1{
	\mathop{\mathrm{\mathstrut \partial}}
	\nolimits^{#1}\gobblespace}
\def\gobblespace{
	\futurelet\diffarg\opspace}
\def\opspace{
	\let\DiffSpace\!
	\ifx\diffarg(
	\let\DiffSpace\relax
	\else
	\ifx\diffarg[
	\let\DiffSpace\relax
	\else
	\ifx\diffarg\{
	\let\DiffSpace\relax
	\fi\fi\fi\DiffSpace}

% derivative and partial derivative operators
\providecommand*{\deriv}[3][]{\frac{\diff^{#1}{#2}}{\diff {#3}^{#1}}}
\providecommand*{\pderiv}[3][]{\frac{\pdiff^{#1}{#2}}{\pdiff {#3}^{#1}}}
\providecommand*{\pderivcross}[3]{\frac{\pdiff^{2} {#1}}{\pdiff {#2} \pdiff {#3} }}

% set notation
\providecommand\given{}
\newcommand\SetSymbol[1][]{%
	\nonscript\:#1\vert
	\allowbreak
	\nonscript\:
	\mathopen{}}
\DeclarePairedDelimiterX\Set[1]\{\}{%
	\renewcommand\given{\SetSymbol[\delimsize]}
	#1
}
\newcommand{\set}[1]{\Set*{#1}}


% probability
\DeclarePairedDelimiterXPP\prob[1]{\mathop{\mathbb{P}}}{\lbrace}{\rbrace}{}{\renewcommand\given{\nonscript\:\delimsize\vert\nonscript\:\mathopen{}}#1}
\newcommand{\Prob}[1]{\prob*{#1}}

\DeclarePairedDelimiterXPP\probability[2]{\mathop{\mathbb{P}}_{#1}}{\lbrace}{\rbrace}{}{\renewcommand\given{\nonscript\:\delimsize\vert\nonscript\:\mathopen{}}#2}
\newcommand{\Probability}[2]{\probability*{#1}{#2}}

\DeclarePairedDelimiterXPP\expectation[1]{\mathop{\mathbb{E}}}{\lbrack}{\rbrack}{}{\renewcommand\given{\nonscript\:\delimsize\vert\nonscript\:\mathopen{}}#1}
\newcommand{\E}[1]{\expectation*{#1}}

\DeclarePairedDelimiterXPP\expectationdist[2]{\mathop{\mathbb{E}}_{#1}}{\lbrack}{\rbrack}{}{\renewcommand\given{\nonscript\:\delimsize\vert\nonscript\:\mathopen{}}#2}
\newcommand{\Exp}[2]{\expectationdist*{#1}{#2}}

\DeclarePairedDelimiterXPP\variance[1]{\mathop{\mathrm{Var}}}{\lbrack}{\rbrack}{}{\renewcommand\given{\nonscript\:\delimsize\vert\nonscript\:\mathopen{}}#1}
\newcommand{\Var}[1]{\variance*{#1}}

\DeclarePairedDelimiterXPP\variancedist[2]{\mathop{\mathrm{Var}}_{#1}}{\lbrack}{\rbrack}{}{\renewcommand\given{\nonscript\:\delimsize\vert\nonscript\:\mathopen{}}#2}
\newcommand{\Variance}[2]{\variancedist*{#1}{#2}}

\DeclarePairedDelimiterXPP\covariance[2]{\mathop{\mathrm{Cov}}}{(}{)}{}{#1,\mathopen{}#2}
\newcommand{\Cov}[2]{\covariance*{#1}{#2}}


% Linear Algebra, symmetry notation
\newcommand{\Matrix}[1]{\begin{bmatrix}#1\end{bmatrix}}
\newcommand{\Tr}[1]{\mathop{\mathrm{Tr}}\squarebrack{#1}}

% norms
\DeclarePairedDelimiterXPP{\nrm}[2]{}{\lVert}{\rVert}{\ensuremath{_{#1}}}{\ifblank{#2}{\:\cdot\:}{#2}}
\newcommand{\norm}[2]{\nrm*{#1}{#2}}

\newcommand{\twonorm}[1]{\norm{2}{#1}}
\newcommand{\twonormsq}[1]{\norm{2}{#1}^2}
\newcommand{\opnorm}[1]{\norm{\mathrm{op}}{#1}}
\newcommand{\normF}[1]{\norm{\mathrm{F}}{#1}}


% parentheses
\DeclarePairedDelimiterX{\bracket}[3]{#1}{#2}{#3}

\newcommand{\abs}[1]{\bracket*{\lvert}{\rvert}{#1}}
\newcommand{\round}[1]{\bracket*{(}{)}{#1}}
\newcommand{\curly}[1]{\bracket*{\lbrace}{\rbrace}{#1}}
\newcommand{\squarebrack}[1]{\bracket*{\lbrack}{\rbrack}{#1}}

% For Convenience
\newcommand{\comment}[1]{\qquad\round{\text{#1}}}
\newcommand{\commentnewline}[1]{\qquad\qquad\qquad\qquad\qquad\qquad\comment{#1}\nonumber}
\newcommand{\inv}[1]{\frac{1}{#1}}
\newcommand{\sumi}[2]{\sum\limits_{i=#1}^{#2}}
\newcommand{\sumj}[2]{\sum\limits_{j=#1}^{#2}}
\newcommand{\sumk}[2]{\sum\limits_{k=#1}^{#2}}

\newcommand{\field}[1]{\mathbb{#1}}
\newcommand{\1}{\mathbf{1}}
\newcommand{\R}{\field{R}} % real domain
\newcommand{\bx}{\mathbf{x}}

% TODO command
\usepackage{xcolor}
\newcommand{\todo}{\textcolor{red!90!black}{\textbf{TODO}}}

\def\diag{\text{diag}}

%% operator in linear algebra, functional analysis
\newcommand{\inner}[2]{#1\cdot #2}
\renewcommand{\theenumi}{\alph{enumi}} 

\newcommand{\Perp}{\perp \! \! \! \perp}

\newcommand\independent{\protect\mathpalette{\protect\independenT}{\perp}}
\def\independenT#1#2{\mathrel{\rlap{$#1#2$}\mkern2mu{#1#2}}}
\newcommand{\vct}[1]{\mathbf{#1}} % vector
\newcommand{\vect}[1]{\mathbf{#1}} % vector
\newcommand{\mat}[1]{\mathbf{#1}} % matrix
\newcommand{\cst}[1]{\mathsf{#1}} % constant
\newcommand{\points}[1]{\small\textcolor{magenta}{\emph{[#1 point\ifthenelse{\equal{#1}{1}}{}{s}]}} \normalsize}

% Define a new problem
\newcounter{problem}
\newenvironment{problem}[1][]{\refstepcounter{problem}\par\medskip\noindent P\theproblem.#1 }

\newcounter{aprob}
\newenvironment{aprob}[1][]{\refstepcounter{aprob}\par\medskip\noindent A\theaprob.#1 }

\newcounter{bprob}
\newenvironment{bprob}[1][]{\begin{mdframed} \refstepcounter{bprob}\par\medskip\noindent B\thebprob.#1 }
  {\vspace{1pt} \end{mdframed}}


\usepackage{listings}  % Include the listings-package
\lstset{language=Python}
\usepackage{float}
\usepackage{multicol}
\usepackage{enumitem}
\usepackage{wrapfig}

\def\P{\mathbb{P}}
\def\R{\mathbb{R}}

\newtheorem{lemma}{Lemma}

\DeclareMathOperator*{\argmax}{arg\,max} 
\DeclareMathOperator*{\argmin}{arg\,min} 
\global\long\def\dd{\textnormal{d}}
