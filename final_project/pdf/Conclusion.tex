\section{Conclusion} \label{Conclusion}
%what is your assessment of the contribution of the paper/report? do you think the assumptions are reasonable? in what ways do you think the work could be complemented or improved? Often, for analysis, there are certain assumption used, for example, small angle approximations, or ignore higher order terms, or assuming the earth is non-rotating, or a particular frame is inertial, etc. In your view, are the corresponding assumptions valid or reasonable? how do you assess/propose that the work be improved?
 Overall the paper heavily depends on the previous works \cite{8270608}, \cite{7322205} and \cite{ALFAKIH2008962}. However the contribution of the paper lies on the fully distributing the followers control strategy for affine formation maneuver and making the leaders maneuver autonomous with given desired trajectory and its finite time derivative information. Also, the PIn type control algorithm does require higher order information than the agents' dynamics, and therefore can not realize the adaptive gain because simply we do not have those information in this setting. Therefore its novelty also lies on usage of different distributed control method for the followers and autonomous tracking control method for leaders. 

 The assumptions here is that the desired trajectory and its finite time derivative information for each agents are accessible and differentiable at nth order. This means that the discontinuous manuever is not available for the leaders and they would be vulnerable sudden change in enviornment. Also the number of leaders are selected at least more than 1 + d in which d is the dimension of the dynamics, that affinely spans $R^d$, which are realizable with 3~4 agents being in the leader group in 2 or 3 dimensions. 
 
 However, another assumption here is that connection between leaders group and the first leader should always exist. Since the continuous maneuver of the first leader and the connection among leaders group are vulnerable to the disturbances, the leaders group should render relatively slow speed for the accurate tracking control algorithm. 
 
 Therefore those assumptions are applicable to the situation where agents are maneuvering through unknown map as it needs slowly maneuver to accurately map the enviornment and avoid any obstacle. Also it has to be applied where the sudden change in enviornment is not expected. For this reason, I suggest, in the future, the research can be improved with conducting performance measure using variance of the disturbance response on this leader-follower affine formation maneuver. Could it be further researched for disturbance rejection control strategy using \cite{8483481}?
