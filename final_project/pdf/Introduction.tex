\section{Introduction}\label{Introduction}

% summary of the contribution of the paper/report/etc. What problem has been addressed, what the proposed analysis/approach aims to solve, why is the problem interesting (theoretically or application-wise). In a nutshell, why should somebody care about the topic (technologically, theoretically, socially (to whatever extend you can see)? 
 
 Due to emerging demand for its potential applications in robotics, aerospace, and automobile industry, multiagent formation control and manuever has been widely studied. In applications such as unmanned ground vehicles, with satillites or with drones, multiagent formation control and manuever algorithm can be designed to maneuver through a unknown map with different designed formation control methods. To design the algorithm behind those controlled formation, it traditionally used agents'relative displacements, distances, or bearing. Although there were different previous works utilizing these traditional methods to solve formation maneuver control problem, they were not able to realize translational, rotational and scaling manuever at the same time. 
 
 To solve this formation maneuver control problem, the paper\cite{9064493}
  utilizes stress matrices among other solutions that realizes those manuevers at the same time, such as berycentric coordinates-based approach and complex Laplacians-based approach, because the berycentric coordinate-based approach requires relative rotation matrix and complex Laplacians-based approach is only applicable in some special dimension, and therefore stress matrix-based approach is more flexible and realizable. In ``Affine Formation Maneuver Control of Multiagent Systems"\cite{8270608}, the formation maneuver control problem is solved with stress matrix-based approach and achieves translational, rotational, scaling, and shearing maneuver concurrently. However, it does not consider the case where the leaders of the formation have time varying acceleration. The followers following leaders with time varying acceleration would have coupled inputs as those followers have to use global information of acceleration of the leaders to calculate control input. 
 
 Therefore the paper suggests two layered leader-follower control strategy with stress matrix to fully distribute the control strategy for followers in affine formation maneuver. Also, for the leaders, it achieves autonomous maneuver control algorithm with information of their desired trajectory and finite time derivatives. 
