\subsection{Manevuer Design} \label{Manevuer Design}
%%Could you verify any of the theoretical steps/simulations- you can be selective here- i want you to show case what you have learned in this class to replicate or redo some of the results/simulations- this section is important as it would allow you to "showoff" what you have learned in 597!
 Before simulating, the desired trajectory has to be generated according to the given desired position and velocity value at time $t_k$ and $t_{k+1}$. With the previous assumption that desired trajectory,$p_i = A(t) r_i + b(t)$, is known and $n^{th}$ order differentiable, the paper assumes $3^{rd}$ order time differentiable equation, 
 \begin{align*}
    b(t) = c_3(t-t_k)^3 + c_2 (t-t_k)^2  + c_1 (t-t_k) + c_0,
 \end{align*}
 where $c_i \in R^{2}$ are constant vectors at $t \in [t_k, t_{k+1}]$ for all the leader positions. With known desired position and velocity profile at two different given time, then the desired trajectory can be found with 
 \begin{align*}
    \begin{bmatrix}
        c_0\\c_1\\c_2\\c_3
    \end{bmatrix}
    = \begin{bmatrix}
        1 & 0 & 0 & 0 \\
        0 & 1 & 0 & 0 \\
        1 & \Delta T & \Delta T ^2 & \Delta T^3 \\
        0 & 1 & \Delta T  & 3\Delta T^2 \\
    \end{bmatrix}^{-1}
    \otimes I_2 
    \begin{bmatrix}
        S_{t_k}\\v_{t_k}\\S_{t_{k+1}}\\v_{t_{k+1}}
    \end{bmatrix}
 \end{align*}
 where $S_{t_k},v_{t_k},S_{t_{k+1}},v_{t_{k+1}}$ are the desired position and velocity profile at $t_k$ and $t_{k+1}$.
 Also the scaling maneuver is designed with, 
 \begin{align*}
    A(t) &= \phi (t) I_2\\
    \phi (t) &=  \varsigma_3(t-t_k)^3 + \varsigma_2 (t-t_k)^2  + \varsigma_1 (t-t_k) + \varsigma_0,
 \end{align*}
 where $\varsigma_3, \varsigma_2 ,\varsigma_1 , \varsigma_0$ are known desired scaling factor at $t_k$ and $t_{k+1}$. With the same matrix equation used for translational maneuver, scaling maneuver constants,$\varsigma_i \in R, i = 0, 1, \cdots 3$, can be also be inferred. 
 Also the rotational and shearing design is described in a same manner, where 
 \begin{align*}
    A_{rot}(t) &= R(\theta(t)) = 
    \begin{bmatrix}
        cos{\theta(t)}&-sin{\theta(t)}\\sin{\theta(t)}&cos{\theta(t)}
    \end{bmatrix}\\
    A_{she}(t) &=
    \begin{bmatrix}
        \psi_1(t)&0\\0&\psi_2(t)
    \end{bmatrix}
 \end{align*}
 are rotational and shearing matrix for each design, and $\theta(t) \in [0,2\pi)$ and $\psi(t)\in (0,1)$ are rotation angle with respect to the nominal configuration and shearing factors respectively. 
 