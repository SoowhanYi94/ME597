\subsection{Simulation and parameters} \label{Simulation and parameters}
%%Could you verify any of the theoretical steps/simulations- you can be selective here- i want you to show case what you have learned in this class to replicate or redo some of the results/simulations- this section is important as it would allow you to "showoff" what you have learned in 597!
 The paper also utilizes MATLAB to calculate the unique solution of the Algebraic Riccati Equation to get values of K.
 \begin{align*}
    K = [46.72 143.817 197.9471 143.817]
 \end{align*}
 Therefore $\eta_i$ is chosen to be 56 bcause $\eta_i \geq \frac{1}{2\lambda_{min}(\Omega_{ff})}$. Other constant parameters are chosen abstract
 \begin{align*}
    \rho_i = 12, \beta = 0.5, k_{i1} = 1, k_{i2} = 16, i = 1, \cdots 4
 \end{align*}
 From figure \ref{paper_simulation} we can see that the paper successfully simulates the designed affine manuever.  
 \begin{figure}[ht] 
    \centering
    \includegraphics*[width=0.4\textwidth]{./img/Paper_simulation.png}
    \caption{Formation maneuver trajectories}
    \label{paper_simulation}
 \end{figure}
  
 Here is my attempt, figure \ref{my_simulation}, to redo the work. It seems like the tracking control algorithm for leaders group is working but it is probably details that I might have missed. But here is my link for the code and future simulation demonstration and effort. \url{https://github.com/SoowhanYi94/ME597}
 \begin{figure}[ht] 
   \centering
   \includegraphics*[width=0.4\textwidth]{./img/Figure_1.png}
   \caption{Formation maneuver trajectories}
   \label{my_simulation}
\end{figure}