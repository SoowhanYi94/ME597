\subsection{Stress matrix Design} \label{Stress matrix Design}
The paper designs the stress matrix according to the methods presented in \cite{7339680} and \cite{8270608}, and the nominal configuration follows,
\begin{align*}
    \begin{matrix}
        r_1 = [0;0] & r_2 = [-1;1] & r_3 = [-1;0] & r_4 = [-1;-1]\\
        r_5 = [-2;1] & r_6 = [-2;-1] & r_7 = [-3;1] & r_8 = [-3;-1]
    \end{matrix},
\end{align*}
and figure \ref{nominal_formation_and_stresses} shows the paper's the nominal configuration and the corresponding weights. 
\begin{figure}[ht]
    \centering
    \includegraphics*[width=0.4\textwidth]{./img/nominal formation and stresses.png}
    \caption{nominal formation and stresses}
    \label{nominal_formation_and_stresses}
\end{figure}