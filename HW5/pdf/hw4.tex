\documentclass{article}
\usepackage{import}
\usepackage[ruled]{algorithm2e}
\usepackage[shortlabels]{enumitem}
\usepackage{hyperref}
\usepackage{minted}
\usepackage{subcaption}

\hypersetup{
    colorlinks=true,
    linkcolor=blue,
    filecolor=magenta,      
    urlcolor=cyan,
    pdftitle={Overleaf Example},
    pdfpagemode=FullScreen,
    }
\subimport*{}{macro}

\setlength\parindent{0px}

\begin{document}
\setcounter{problem}{0}
\title{Homework \#4}
\author{
    \normalsize{AA 597: Networked Dynamics Systems}\\
    \normalsize{Prof. Mehran Mesbahi}\\
    \normalsize{Due: Feb 9, 2024 11:59pm}\\
    \normalsize{Soowhan Yi}
}
\date{{}}
\maketitle

All the codes are available at the end of the documents or here.
\url{https://github.com/SoowhanYi94/ME597}
\begin{problem}7.1
    This chapter mainly dealt with $\Delta$-disk graphs, that is, proximity graph (V,E) such that ${v_i, v_j} \in E$ if and only if $||x_i - x_j|| \leq \Delta$. wjere $x_i \in R^p, i = 1, \codts, n, $ is the state of robot i. In this exercise, we will be exploring another type of proximity graphy, namely the wedge graph. Assume that instead of single integrator dynamics, the agents' dynamics are defined as unicycle robots, that is, 
    \begin{align*}
        &\dot x_i(t) = v_i(t) cos{\phi_i (t)}\\
        &\dot y_i(t) = v_i(t) sin{\phi_i (t)}\\
        &\dot \phi_i(t)= \omega_i(t)
    \end{align*}
    
    Here $[x_i, y_i]^T$ is the position of robot i, while, $\phi_i$ denotes its orientation. Moreover, $v_i$ and $\omega_i$ are the translational and rotational velocities, which are the controlled inputs. Now, assume that such a robot is equiopped with a rigidly mounted camer, facing in the forward direction. This gives rise to a directed wedge graph, as seen in the figure. For such a setup, if robot j is visible from robot i, the available information is $d_{ij} = ||[x_i, y_i]^T - [x_j, y_j]^T||$ (distance between agents) and $\delta \phi_{ij}$(relative interagent angle) as per the figure below. Explain how you would solve the rendezvous (agreement) problem for such a system. 
    
    \vspace{12pt}
    In order to solve this agreement problem, we need to controll the distance between two agents, especially between the leaf nodes($v_j$ and $v_k$ in the example) and the center node($v_i$), and angle between them. So we need to use combination of unicycle control law and control law for single integrator. 
      
    Using Laplacian-based protetial, 
    \begin{align*}
        W_m(\theta) = \frac{1}{2} (e^{jm\theta}) * L(G)(e^{jm\theta}) 
    \end{align*}
\end{problem}

\end{document}