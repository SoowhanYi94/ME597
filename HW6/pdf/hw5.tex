\documentclass{article}
\usepackage{import}
\usepackage[ruled]{algorithm2e}
\usepackage[shortlabels]{enumitem}
\usepackage{hyperref}
\usepackage{minted}
\usepackage{subcaption}
\hypersetup{
    colorlinks=true,
    linkcolor=blue,
    filecolor=magenta,      
    urlcolor=cyan,
    pdftitle={Overleaf Example},
    pdfpagemode=FullScreen,
    }
\subimport*{}{macro}

\setlength\parindent{0px}

\begin{document}
\setcounter{problem}{0}
\title{Homework \#6}
\author{
    \normalsize{AA 597: Networked Dynamics Systems}\\
    \normalsize{Prof. Mehran Mesbahi}\\
    \normalsize{Due: Mar 1, 2024 11:59pm}\\
    \normalsize{Soowhan Yi}
}
\date{{}}
\maketitle

All the codes are available at the end of the documents or here.
\url{https://github.com/SoowhanYi94/ME597}
\begin{problem}3.8
    Consider the uniformly delayed agreement dynamics over a weighted graph, specified as
    \begin{align*}
        \dot x_i(t) = \sum_{j \in N(i)} w_{ij} (x_j(t - \tau) - x_i (t-\tau))
    \end{align*}
    for some $\tau > 0$ and $i = 1, \cdots n$. Show that this delayed protocol is stable if 
    \begin{align*}
        \tau < \frac{\pi}{2\lambda_n(G)}
    \end{align*}
    where $\lambda_n(G)$ is the largest eigenvalue of the corresponding weighted Laplacian. Conclude that, for the delayed agreement protocol, there is a tradeoff between faster convergence rate and tolerance to uniform delays on the information-exchange links.
\end{problem}
\begin{problem} 3.9
    A matrix M is called essentially non-negative if there exists a sufficiently large $\mu$ such that $M + \mu I$ is non-negative, that is, all its entries are non-negative. Show that $e^{tM}$ for an essentially non-negative matrix M is non-negative when $t \geq 0.$

\end{problem}
\begin{problem} 3.16
    Consider a network of n processors, where each processor has been given an initial computational load to process. However, before the actual processing occurs, the processors go through an initialization phase, where they exchange certain fractions of their loads with their neighbors in the network. Specifically, during this phase, processor i adopts the load-update protocol
\end{problem}

\begin{problem} 8.1
    Let $H_i , i = 1, 2, 3$, be the rows of the $3 \times 3$ identity matrix in the observation scheme $z_i = H_i$ $x + v_i$ for a three-node sensor network, observing state $x \in R^3$ . It is assumed that the nodes form a path graph and that $v_i$ is a zero-mean, unit variance, Gaussian noise. Choose the weighting matrix W (8.8) and the step size $\Delta$ in (8.20) - (8.21), conforming to the condition (8.14). Experiment with the selection of the weights for a given value of $\Delta$ and their effect on the convergence properties of the distributed least square estimation (8.20) - (8.21).
\end{problem}
\begin{problem} 10.8
    If the network is connected, then the followers will end up (asymptotically) at
    \begin{align*}
        x_f = -A_f^{-1} B_f x_l
    \end{align*}
    given the static leader positions $x_l$ . Show that each component of $x_f$ above is in fact given by a convex combination of the components of $x_l$ .
\end{problem}
\begin{problem} 10.9
    Consider the linear-quadratic optimal control problem
    \begin{align*}
        \min_{u} \int_{0}^{\infty} (u(t)^T R u(t) + x(t)^T Q x(t))dt
    \end{align*}
    where the matrices Q and R are, respectively, positive semidefinite and positive definite, and 
    \begin{align*}
        \dot x(t) = -A_f x(t) - B_f u(t) 
    \end{align*}
    corresponds to a controllable network.
\end{problem}

\end{document}